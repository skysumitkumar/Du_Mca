\documentclass[english]{article} 

\usepackage[english]{babel}
\usepackage{times}
\usepackage{algorithmic}
\usepackage{algorithm}
\usepackage{fancyhdr}

\begin{document}

\pagenumbering{arabic}

%----Header stuff---------------------------
\pagestyle{fancy}
 \lhead{\sc Algorithms \& Data Structures}
 \chead{}
 \rhead{\sc INF1 -- 2003}

\section*{Tutorial 3}

\subsection*{Exercise 1} 

Let $T_j(n)$, such that $1 \leq j \leq k$, be the maximum
of $T_1(n), \ldots, T_k(n)$ and recall that the probabilities
$p_i$ satisfy $p_1 + p_2 + \cdots + p_k = 1$. Then

$$A(n) = 
\sum_{i = 1}^k p_i\cdot T_i(n) \leq
\sum_{i = 1}^k p_i\cdot T_j(n) \leq
T_j(n) \sum_{i = 1}^k p_i = T_j(n) = \max_{1 \leq i \leq k} T_i(n) = W(n).$$ 

\subsection*{Exercise 2}

We shall solve the following recurrence equation
\begin{center}
\begin{tabular}{rcl}
$T(1)$ & $=$ & $7$ \\
$T(n)$ & $=$ & $2 + T(n-1)$ for $n>1$  
\end{tabular}
\end{center}
by repeated substitutions.
Let us unfold $T(n)$. 
$$T(n) = 2 + T(n-1) = 2 + 2 + T(n-2) = 2 + 2 + 2 + T(n-3) = \ldots $$
Observe the pattern 
$T(n) = \underbrace{2 + 2 + \cdots + 2}_{i \times} + T(n-i)$
for $i < n$. So if $i = n -1 $ then
$$T(n) = \underbrace{2 + 2 + \cdots + 2}_{(n-1) \times} + T(n-n+1)=
2(n-1) + T(1) = 2n - 2 + 7 = 2n +5.$$
Proof: by induction on n we show that $T(n)=2n + 5$. In the base case
$n=1$ and $T(1)=2\cdot 1 + 5 = 7$ as required. Let $n>1$.
By IH (induction hypothesis) assume that $T(j) = 2j +5$ for all
$1 \leq j < n$. Now $$T(n) = \mbox{(by definition) } 2 + T(n-1) =
\mbox{(by IH) }
2 + 2(n-1) + 5 = 2 + 2n - 2 + 5 = 2n + 5.$$

\subsection*{Exercise 3}

We shall solve the following recurrence equation
\begin{center}
\begin{tabular}{rcl}
$T(0)$ & $=$ & $1$ \\
$T(n)$ & $=$ & $nT(n-1)$ for $n>1$  
\end{tabular}
\end{center}
by repeated substitutions.
Let us unfold $T(n)$. 
$$T(n) = nT(n-1) = n(n-1)T(n-2) = n(n-1)(n-2)T(n-3) = ... $$
Observe the pattern
$T(n) = n(n-1)(n-2)\cdots(n-i+1)T(n-i)$
for $i \leq n$. So if $i = n$ then
$$T(n) = n(n-1)(n-2)\cdots(n-n+1)T(0)=
 n(n-1)(n-2)\cdots 2\cdot 1\cdots T(0) = n!.$$

Proof: by induction on n we show that $T(n)=n!$. In the base case
$n=0$ and $T(0)=1=0!$ as required. Let $n>0$.
By IH (induction hypothesis) assume that $T(j) = j!$ for all
$0 \leq j < n$. Now $$T(n) = \mbox{(by definition) } nT(n-1) =
\mbox{(by IH) }
n\cdot (n-1)! = n!.$$

\subsection*{Exercise 4}

Let $n$ be the size of the array, i.e., $n = b-a+1$ and assume
that $a \leq b$.
Then $W(1)= 1$ and $W(n)=1 + W( \lfloor (n+1)/2 \rfloor)$ for $n >1$.
Let us reformulate the recurrence equation such that we consider
only the points where $n = 2^k$ for some integer $k \geq 0$.

Let $n = 2^k$. Observe that $\lfloor (n+1)/2 \rfloor =
\lfloor (2^k+1)/2 \rfloor = 2^{k-1}$.
Hence we shall solve the following recurrence equation
\begin{center}
\begin{tabular}{rcl}
$W(2^0)$ & $=$ & $1$ \\
$W(2^k)$ & $=$ & $1 + W(2^{k-1})$ for $k > 0$. 
\end{tabular}
\end{center}
by repeated substitutions.
Let us unfold $W(2^k)$.
$$W(2^k) = 1 + W(2^{k-1}) =
 1 + 1 + W(2^{k-2}) = 
 1 + 1 + 1 + W(2^{k-3}) = \ldots$$
Observe the pattern
$W(2^k) = i + W(2^{k-i})$
for $i \leq k$. So if $i = k$ then
$$W(2^k) =  k + W(2^0) = k + 1.$$
However, $2^k = n$ implies that $k = \log_{2}(n)$.
So $W(n) = \log_{2}n + 1$.

Proof: by induction on k we show that $W(2^k)=k+1$. 
In the base case
$k=0$ and $W(2^0) = \mbox{ (by definition) }1 = 0+1$ 
as required. Let $k>0$.
By IH (induction hypothesis) assume that $W(2^j) =j+1$ for all
$0 \leq j < k$. Now $$W(2^k) = \mbox{(by definition) } 1 +  
W(2^{k-1}) = \mbox{(by IH) }
1 + (k-1) + 1 = k+1.$$

\subsection*{Exercise 5} 

$$T(n) = n^3$$

\subsection*{Exercise 6} 

\begin{itemize}
\item[a)] Let us fix $n_0=1$ and $M=5$, then
  $n+4 \leq 5n$ for all $n\geq 1$. Another
  possibility is e.g. $n_0=4$ and $M=2$ because
  $n+4 \leq 2n$ for all $n \geq 4$. The validity
  of the second case can be proved by induction.
  In the basic step we have to check that  $4+4 \leq 2\cdot 4$
  which is true. For the inductive step
  let $n>4$ and by IH we assume 
  that $(n-1) + 4 \leq 2(n-1)$. We want to show that
  $n + 4 \leq 2n$, which is true because
  $$n+4 = (n-1) + 4 + 1 \leq \mbox{ (by IH) }
  2(n-1) +1 = 2n - 2 + 1 = 2n -1 \leq 2n.$$
 
\item[b)] Let us fix e.g. $n_0 = 1$ and $M=5$, then
$3n^5 + 2n^3 \leq M\cdot n^5$ for every $n \geq n_0$.
Why and how did we find it?
$$3n^5 + 2n^3 \leq 3n^5 + 2n^5 = 5n^5$$

\item[c)] Let $n_0=100$ and $M=1$. Then $n^7 \leq 2^n$ for all 
$n \geq 100$. The argument for ``why'' is a little bit more difficult
this time and we will prove it by induction on $n$.

For the base case (n=100) we have 
$100^7 \leq 2^{100}$ which is true (check this on a calculator if you want
 to :-)).

Let $n>100$ and we assume by IH that $(n-1)^7 \leq 2^{n-1}$.
We aim to show that 
\begin{equation} \label{eq1}
n^7 \leq 2^n.
\end{equation}

In order to do that we will in fact prove a stronger claim 
\begin{equation} \label{eq2}
n^7 \leq (n-1)^7\cdot 2.
\end{equation}

If we succeed  to show this then by IH 
$(n-1)^7\cdot 2 \leq 2^{n-1}\cdot 2 = 2^n$, and
the inequality (\ref{eq1}) is also proven.

To finish the proof we will now argue that (\ref{eq2}) is true
for all $n > 100$.
To prove $n^7 \leq (n-1)^7\cdot 2$ is the same as to prove
$\sqrt[7]{n^7} \leq \sqrt[7]{(n-1)^7\cdot 2}$
(both sides are non-negative), which is the
same as $n \leq (n-1)\cdot  \sqrt[7]{2}$, which is the same
as $n \leq n\cdot  \sqrt[7]{2} - \sqrt[7]{2}$, which is the same as
$\sqrt[7]{2} \leq n\cdot (\sqrt[7]{2} - 1)$, which is the same as
$n \geq \frac{\sqrt[7]{2}}{\sqrt[7]{2} - 1}$, which is obviously true
for all $n > 100$ because $\frac{\sqrt[7]{2}}{\sqrt[7]{2} - 1} \simeq 11$.



\item[d)] Let us fix. e.g. $n_0 = 2$ and $M=1/2$, then
      $n^2\cdot \log n \geq n^2 = (1/2)\cdot 2 n^2$ 
      for all $n \geq 2$.

\item[e)] First we show that $5n^2 + 2n + 4$ is $O(n^2)$.
      Let $n_0=1$ and $M=11$. Then  
  $$5n^2 + 2n +4 \leq 5n^2 + 2n^2 + 4n^2 = 11n^2 =
 M\cdot n^2$$
  for all $n \geq 1$. \\
      Next we have to show that $5n^2 +2n +4$ is
    $\Omega(n^2)$. Let $n_0=1$ and $M = 1$ which immediately
    implies that $5n^2 +2n +4 \geq n^2$ for all $n \geq 1$.

\end{itemize}

\subsection*{Exercise 7}

\begin{itemize}
\item $20n^2 = O(n^2)$ \\
(!!! {\bf but also} e.g. $20n^2 = O(n^3)$, $20n^2 = O(n^8)$ and
  $20n^2 = O(2^n)$ even if these are not optimal estimates)
\item $10n^3 + 6n = O(n^3)$
\item $2^n + n^{1691} = O(2^n)$
\item $1000n + n^2 = O(n^2)$
\item $\log_5(3n^4) = O(\log_2 n)$\\
  (and we usually write only $O(\log n)$
  since the base of the logarithm is irrelevant for the $O$-notation
  as proved during the lecture) \\
  The argument for this is the following:
   $\log_5{(3n^4)} = \log_5{3} + \log_5{n^4} = 
  \log_5{3} + 4\cdot\log_5 n = O(\log_5{n}) = O(\log_2{n})$
\end{itemize}

\subsection*{Exercise 8} 

It is true that $2^n$ is $O(3^n)$ but $2^n$ is not $\Omega(3^n)$.
Hence $2^n$ is not $\Theta(3^n)$. \\

\subsection*{Exercise 9}

Assume that $f(n)=O(g(n))$ and $g(n)=O(h(n))$ which means that
there are natural numbers $n_0$ and $n_1$, and real numbers
$M_0$ and $M_1$ such that 
\begin{itemize}
\item
$f(n) \leq M_0\cdot g(n)$ for all $n \geq n_0$ and
\item
$g(n) \leq M_1\cdot h(n)$ for all $n \geq n_1$.
\end{itemize}
Let us define $n_2 = \max\{n_0,n_1\}$ and $M_2=M_0\cdot M_1$,
which implies that
$$f(n) \leq M_0\cdot g(n) \leq M_0\cdot ( M_1\cdot h(n)) =
(M_0\cdot M_1)\cdot h(n) = M_2\cdot h(n)$$ for all
$n \geq n_2$ and hence $f(n)=O(h(n))$.

\end{document}
