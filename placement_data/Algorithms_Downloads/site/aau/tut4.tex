\documentclass[english]{article} 

\usepackage[english]{babel}
\usepackage{times}
\usepackage{algorithmic}
\usepackage{algorithm}
\usepackage{fancyhdr}

\begin{document}

\pagenumbering{arabic}

%----Header stuff---------------------------
\pagestyle{fancy}
 \lhead{\sc Algorithms \& Data Structures}
 \chead{}
 \rhead{\sc INF1 -- 2003}


\section*{Tutorial 4}

\subsection*{Example 1}
\begin{itemize}
\item The space/memory requirements ($S(n)$)
 of an algorithm is usually measured by investigating the size of the most expensive data structure used (how many memory cells are needed to store it).
The characteristic data structure of the given algorithm fib$(n)$ is
{\tt table}. 
% {\tt table} is initiated with values for {\tt table[0]} and {\tt table[1]} and is added an element for each $n>2$ additionally the $i$ variable used in the {\tt FOR} construct uses a constant size.
In a call fib$(n)$ the size of the table is
$n+1$ hence $S(n)$ is $O(n)$.
\item
\begin{algorithmic}
  \STATE{{\bf fib}($n$) =}
  \IF {$n=0$}       
   \STATE{$fib:=0$}
  \ELSE
   \STATE{$fib:=1$}
  \ENDIF
  \STATE{$fib_{-1}:=1$}
  \STATE{$fib_{-2}:=0$}
  \FOR{$i=2$ to $n$}
    \STATE{$fib := fib_{-1} + fib_{-2}$}
	\STATE{$ fib_{-2}:= fib_{-1} $}
	\STATE{$fib_{-1}:= fib$}
  \ENDFOR
  \STATE{{\bf return} $fib$}
\end{algorithmic}
\end{itemize}

\subsection*{Example 2}
\begin{description}
\item[Pre:] $A[a..b]$ is an array such that $a\leq b$
\item[] res:= largest$(a,b)$
\item[Post:] res $\geq A[a..b]$ and res $\in A[a..b]$ and $A'=A$
\\[4mm]
\begin{algorithmic}
  \STATE{{\bf largest}($a$,$b$) =}
  \IF{$a=b$}
    \STATE{$res := A[a]$}
  \ELSE
    \STATE{$mid:= \lfloor \frac{a+b}{2} \rfloor$}
        \STATE{$sol_1 := \mbox{\bf largest}(a,mid)$}
        \STATE{$sol_2 := \mbox{\bf largest}(mid+1,b)$}
	\STATE{$res:=${\bf max}$(sol_1,sol_2)$} 
  \ENDIF
  \STATE{{\bf return} $res$}
\end{algorithmic}
\end{description}

\newpage
\subsection*{Example 3}
\begin{itemize}
\item Let $n$ be the size of the considered array, i.e., $n = b-a+1$.
\item[] $W(1)=1$
\item[] $W(n)= 1 + 2W(\lceil \frac{n}{2} \rceil)$ \ \ \ for $n>1$
\item We solve the equation only in points where $n=2^k$
      for all $k = 0, 1, 2, \ldots$. Hence we consider the following 
      recurrence equation.
     
\item[] $W(2^0)=1$
\item[] $W(2^k)= 1 + 2W(2^{k-1})$ \ \ \ for $k>0$

\item We shall use repeated substitutions to solve this system.
 \begin{eqnarray*}
      W(n) = W(2^k) &=& 1 +2W(2^{k-1}) \\
			    &=& 1 +2(1+2W(2^{k-2}))\\
				&=& 1 +2 +4W(2^{k-2})\\
				&=& 2^0 + 2^1 + \ldots + 2^{i}W(2^{k-i}), \ \ \
(0\leq i\leq k)\\
				&=& 2^0 + 2^1 + \ldots + 2^{k}\underbrace{W(2^{k-k})}_{= W(2^0) = 1}, \ \ \
(i=k) \\
				&=& 2^0 + 2^1 + \ldots + 2^{k} \\
				&=& 2^{k+1}-1\\
				&=& 2^{(\log n)+1}-1\\
				&=& 2 \cdot 2^{\log n} - 1\\
				&=& 2n-1\\
				&=& O(n)
        \end{eqnarray*}
\end{itemize}

\subsection*{Example 4}
\begin{algorithmic}
 \STATE{$i:=1$}
 \STATE{$j:=2$}
 \WHILE{$max(i,j)<n$}
   \IF{$i$ knows $j$}
     \STATE{$i:=max(i,j)+1$}
   \ELSE
     \STATE{$j:=max(i,j)+1$}
   \ENDIF
 \ENDWHILE
 \STATE{$candidate := min(i,j)$}
 \STATE{}
 \STATE{*** Now we check whether $candidate$ is indeed a celebrity ***}
 \STATE{}
  \STATE{OK:=true; \  $i:=1$}
  \WHILE {OK and $i\leq n$} 
  \IF {$candidate \not= i$}
  \IF {$candidate$ knows $i$} 
   \STATE{OK := false} 
  \ELSE \IF {$i$ does not know $candidate$} 
   \STATE{OK := false} 
  \ENDIF
  \ENDIF
  \ENDIF
  \STATE{i:=i+1}
  \ENDWHILE 
  \IF {OK=true} 
   \STATE{{\bf return} "$candidate$ is a celebrity"}
  \ELSE
   \STATE{{\bf return} "there is no celebrity"}
  \ENDIF
\end{algorithmic}

\end{document}
