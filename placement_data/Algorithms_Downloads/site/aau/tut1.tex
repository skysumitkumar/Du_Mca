\documentclass[english,twoside]{article} 

\usepackage[english]{babel}
\usepackage{times}
\usepackage{algorithmic}
\usepackage{algorithm}
\usepackage{fancyhdr}

\begin{document}

\pagenumbering{arabic}

%----Header stuff---------------------------
\pagestyle{fancy}
 \lhead{\sc Algorithms \& Data Structures}
 \chead{}
 \rhead{\sc INF1 2004}


\section*{Tutorial 1}

\subsection*{Example 1}

\begin{algorithmic}
\STATE $x := 1$
\REPEAT
\STATE $x := x*2$
\UNTIL (NOT $x<20$)
\STATE $x := x+1$

\bigskip

Similarly, the same can be achieved also by using a {\bf while}-loop.

\end{algorithmic}


\subsection*{Example 2}

{\bf INPUT:} A sequence $S$ of natural numbers of length $n$ such that
     $n > 0$. \\
{\bf OUTPUT:} The smallest and largest element of $S$.\\

\begin{itemize}
\item
 \begin{algorithmic}
   \STATE $min := S[1]$; $max := S[1]$
   \FOR {$i:=2$ to $n$}
      \IF {$S[i]< min$}
         \STATE {$min:=S[i]$}
	  \ELSE \IF {$S[i] > max$}
	     \STATE {$max:=S[i]$}
		 \ENDIF
	  \ENDIF
   \ENDFOR
   \STATE {\bf return} $min$, $max$
 \end{algorithmic}

\item In the worst-case the number of comparisons is $2(n-1)=2n - 2$.
\item Yes. For example let $n=2$
 and let $S=[3,4]$ (i.e. $S[1]=3$ and $S[2]=4$). Then
  the algorithm performs $2$ comparisons.
  If $S=[6,5]$ (i.e. $S[1]=6$ and $S[2]=5$) the algorithm 
  performs only $1$ comparison. \\
{\bf Remark:} For your algorithm the answer can be different.
\end{itemize}

\subsection*{Example 3}

{\bf INPUT:} A sequence $S$ of natural numbers of length $n$ such that
$n > 0$. \\
{\bf OUTPUT:} The second smallest ($ssmall$) element of $S$.\\

\begin{itemize}
\item
 \begin{algorithmic}
   \STATE $min :=  \infty$; $ssmall := \infty$
   \FOR{$i:=1$ to $n$}
     \IF {$S[i] < ssmall$}
	   \IF {$S[i] < min$}
	     \STATE {$ssmall := min$}
		 \STATE {$min := S[i]$}
	   \ELSE
	     \STATE {$ssmall := S[i]$}
	   \ENDIF
	 \ENDIF  
   \ENDFOR
   \IF {$ssmall = \infty$}
     \STATE {{\bf return} "second smallest does not exist"}
   \ELSE
     \STATE {{\bf return} $ssmall$}
   \ENDIF
 \end{algorithmic}

 
\item Number of comparisons in the worst-case is $2n+1$.
\item Yes.
\item For each $n$, the algorithm performs the largest number of comparisons
on e.g. the following sequence: $S=[n, n-1, n-2, \ldots, 2, 1]$
 (i.e. $S[1]=n$, $S[2]=n-1$, $S[3]=n-2$, \ldots, $S[n-1]=2$, $S[n]=1$).
\end{itemize}

\subsection*{Example 4}

{\bf INPUT:} Text  $x$ of length $n$ and pattern $y$ of length $m$ such that
 $n > m > 0$.\\
{\bf OUTPUT:} Number of occurrences of $y$ in $x$.\\

\begin{itemize}
\item
  \begin{algorithmic}
  \STATE {$occurrences:=0$}
  \FOR {$i:=1$ to $n-m+1$}
    \STATE {$j:=1$}
    \WHILE {$x[i+j-1] = y[j]$ and $j \leq m$}
		  \STATE {$j:=j+1$}
	\ENDWHILE
	\IF {$j=m+1$} \STATE  {$occurrences:=occurrences+1$} 
        \ENDIF
  \ENDFOR
  \STATE {{\bf return} $occurrences$}
  \end{algorithmic}

\item Number of comparisons in the worst-case (we count only the comparisons
between two positions in $x$ and $y$): 
$(n-m+1)\cdot m = nm - m^2 + m = O(nm)$ because $n>m$.

\item For each $n$ and $m$,  the largest number of comparisons 
will be e.g. on the input $x=[\underbrace{A, A,\ldots, A}_{n \times}]$
and $y=[\underbrace{A, A, \ldots, A}_{m \times}]$.
\end{itemize}

\end{document}
