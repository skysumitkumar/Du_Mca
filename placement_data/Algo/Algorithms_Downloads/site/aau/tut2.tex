\documentclass[english]{article} 

\usepackage[english]{babel}
\usepackage{times}
\usepackage{algorithmic}
\usepackage{algorithm}
\usepackage{fancyhdr}

\begin{document}

\pagenumbering{arabic}

%----Header stuff---------------------------
\pagestyle{fancy}
 \lhead{\sc Algorithms \& Data Structures}
 \chead{}
 \rhead{\sc INF1 -- 2003}


\section*{Tutorial 2}

\subsection*{Exercise 1}
(Kingston, exercise 1.1, page 11).
\begin{algorithmic}
 \STATE {${\bf g(n)}\lbrace$}
 \IF {$n \leq 1$}
   \STATE {$Result:=n$}
 \ELSE
   \STATE {$Result:= 5 \times g(n-1) - 6 \times g(n-2)$}
 \ENDIF
 \STATE{$\rbrace$}
\end{algorithmic} 
\begin{description}
  \item[Claim:] For all integers $n \geq 0$ it holds that
{\bf g}$(n)=3^n - 2^n$ (i.e. for all $n$ greater than or equal to $0$
 the function {\bf g}($n$) returns $3^n - 2^n$).
  \item[Proof:] The proof is by induction on $n$.
  \begin{description}
    \item[Basic step:] Let $n=0$. Then
    $g(0)$ returns $0$ and $3^0-2^0=0$ $\surd$.\\
 Let $n=1$. Then $g(1)$ returns $1$ and $3^1-2^1=1$ $\surd$.
	\item[Inductive step:] Let $n>1$. IH=''$g(j)$ returns $3^j-2^j$ for all 
integers $j$ in the range $0 \leq j \leq n-1$'', which is in accordance with the second principle of mathematical induction (see Rosen p. 196 available in the course folder). From the induction hypothesis (IH), we can assume that $g(n-1)$ returns $3^{n-1}-2^{n-1}$ and that  $g(n-2)$ returns $3^{n-2}-2^{n-2}$. Now we show that the value returned by the function $g(n)$ equals $3^n - 2^n$ under this assumption. Because $n>1$ the test of the {\tt IF} statement is false, thus $g(n)$ returns $5 \cdot g(n-1)-6\cdot g(n-2)$, and
	  \begin{eqnarray*}	  
	  5 \cdot g(n-1) - 6 \cdot g(n-2) &\stackrel{\mbox{IH}}{=}& 5 \cdot (3^{n-1}-2^{n-1}) - 6 \cdot (3^{n-2}-2^{n-2}) \\
	  &=& 5 \cdot (3 \cdot 3^{n-2} - 2 \cdot 2^{n-2}) - 6 \cdot (3^{n-2}-2^{n-2}) \\
	  &=& 15 \cdot 3^{n-2} - 10 \cdot 2^{n-2} - 6 \cdot 3^{n-2} + 6 \cdot 2^{n-2} \\
	  &=& 9 \cdot 3^{n-2} - 4 \cdot 2^{n-2} \\
	  &=& 3^2 \cdot 3^{n-2} - 2^2 \cdot 2^{n-2} \\
	  &=& 3^n - 2^n \surd
	  \end{eqnarray*}
and this is indeed equal to $3^n - 2^n$ as required. 
  \end{description} 
\end{description}

\newpage

\subsection*{Exercise 2}
\begin{description}
  \item[Pre:] $X=a,Y=b$.
  \item[Post:] $X=b,Y=a$.
  \item[Proof:] The proof is performed by the assertion method:
  \begin{itemize}
    \item[] \hspace{1cm} {\scriptsize $\{X=a,Y=b\}$}
	\item[-]$X:=X+Y$
	\item[]\hspace{1cm} {\scriptsize $\{X=a+b,Y=b\}$}
	\item[-]$Y:=X-Y$
	\item[]\hspace{1cm} {\scriptsize $\{X=a+b,Y=a\}$}
	\item[-]$X:=X-Y$
	\item[]\hspace{1cm} {\scriptsize $\{X=b,Y=a\}$}
  \end{itemize}
\end{description}

\subsection*{Exercise 3}
\begin{itemize}
\item No, the algorithm is not totally correct, e.g. $-2$ satisfies the 
pre-condition, however the algorithm does not terminate on this input.
\item Yes. Let $n$ be an arbitrary integer.
If $n<0$ then the algorithm does not terminate and hence no post-condition
has to be checked.
If $n\geq 0$ then the algorithm terminates and outputs $n!$ which satisfies
the post-condition.
\end{itemize}

\subsection*{Exercise 4}
\begin{description}
  \item[Claim:] The specification:
    \begin{description}
      \item[pre:] $a \leq b+1$
	  \item[post:] $entries.item(a) \leq entries.item(a+1)\leq \ldots \leq entries.item(b)$
    \end{description}
    is satisfied by the algorithm of exercise 1.2 on page 11.
  \item[Proof:] The proof is by induction on $n$ (the length of the array,
 i.e. $n=b-a+1$).
We will prove the claim for a stronger post-condition
post' (this will of course imply that also post is satisfied):

  \item[post':] ``$selection\_sort(a,b)$ changes only the
values between $a$ and $b$ and the changed values are some permutation
of the original ones such that
 $entries.item(a) \leq entries.item(a+1)\leq \ldots \leq entries.item(b)$''
  \begin{description}
  \item[Basic step:] Let $n=0$.
 This implies that $a=b+1$ (the array is empty and hence sorted)
  and the algorithm does nothing as required $\surd$.
    %\item[Basis step:]  $i=a$ and $j=b$. Given the initial assumptions $entries.item(a)$ contains the minimal element of the array.
	\item[Inductive step:] Let $n \geq 1$. 

IH=``For all $j$, $0 \leq j \leq n-1$, and for all $a$ and $b$ such that
$j=b-a+1$ it holds that
$selection\_sort(a,b)$ changes only the
values between $a$ and $b$ and the changed values are some permutation
of the original ones such that
 $entries.item(a) \leq entries.item(a+1)\leq \ldots \leq entries.item(b)$''

Let us now consider a call of $selection\_sort(a,b)$ such 
such that $b-a+1 = n$. We want to show that after its execution
the condition post' will be true.

From IH we can assume that $selection\_sort(a+1,b)$ sorts the entries 
between $a+1$ and $b$ ($b - (a+1) +1 < b - a +1$) such that
$entries.item(a+1) \leq entries.item(a+2)\leq \ldots \leq entries.item(b)$ and nothing else is changed.

In the call of $selection\_sort(a,b)$ 
the else branch of the if-command is
executed ($n\geq 1$) and
the entries $a+1$ to $b$ are sorted by the recursive call
$selection\_sort(a+1,b)$ (hence by the IH: $entries.item(a+1) \leq entries.item(a+2)\leq \ldots \leq entries.item(b)$)
and we also know that 
$entries.item(a) \leq entires.item(i)$ for all $i$, $a \leq i \leq b$
(this holds after performing min\_index and swap), hence in particular
also $entries.item(a)$\\$ \leq entries.item(a+1)$. This means that post'
is satisfied  $\surd$.
\end{description}
\end{description}

\end{document}
