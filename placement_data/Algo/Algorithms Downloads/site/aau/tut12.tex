\documentclass[english]{article} 

\usepackage[english]{babel}
\usepackage{times}
\usepackage{algorithmic}
\usepackage{algorithm}
\usepackage{fancyhdr}
\usepackage{color}
\usepackage[all]{xypic}

\begin{document}

\pagenumbering{arabic}

%----Header stuff---------------------------
\pagestyle{fancy}
 \lhead{\sc Algorithms \& Data Structures}
 \chead{}
 \rhead{\sc INF1 -- 2003}



\section*{Tutorial 12}
\subsection*{Exercise 1}
\begin{itemize}
\item $1$
\item $4$
\item $9$
\item $25$
\item $n^2$
\end{itemize}

\begin{tabular}{l|ll}
 & Adjacency Matrix & Adjacency List \\
 \hline
Nothing known & $O(n^2)$ & $O(n+m)$ \\
$m= O(n)$ & $O(n^2)$ & $O(n)$ \\
$m=\Omega(n^2)$ & $O(n^2)$ & $O(n^2)$ \\
$n=\Omega(m^2)$ & $O(n^2)$ & $O(n)$ \\
$m=O(n \cdot \log n)$ & $O(n^2)$ & $O(n \cdot \log n)$\\
\end{tabular}

\begin{itemize}
\item
Adjacency list is more space efficient, the 2nd, 4th and 5th row 
from the table above demonstrate this. If the graph is sparse
(only a few edges) then adjacency list is much more space efficient
compared to adjacency matrix. On the other hand if the graph contains
almost all possible edges, the space complexity of both implementations
is asymptotically the same. 
 \end{itemize}
\subsection*{Exercise 2}

\begin{itemize}
\item  \mbox{}
  \xymatrix@R=5ex@C=4ex{
A\ar[r] & B \ar[d] & C \ar[l] \ar[d]\\
& E\ar[r] & D
  }

One sequence of topologically ordered nodes of the DAG G 
      is e.g  A, C, B, E, D.
\item

The following DAG has exactly one topologically ordered sequence of nodes. \\[3mm]

  \xymatrix@R=5ex@C=4ex{
A\ar[r] & B \ar[r] & C \ar[r] & D \ar[r] & E
  }
\mbox{} \\[3mm]

This one has also exactly one topologically ordered sequence of nodes
but contains more edges. \\[3mm]

  \xymatrix@R=5ex@C=4ex{
A\ar[r] \ar@/^1pc/[rr] & B \ar[r] & C \ar[r] \ar@/^1pc/[r]& D \ar[r] & E
  }
\mbox{} \\[3mm]
\item Consider e.g. the following DAG.  \\[3mm]

 \xymatrix@R=2ex@C=2ex{
A \ar[rd] & & D \\
& C \ar[ru] \ar[rd] \\
B \ar[ru] & & E
  }
\mbox{}\\[3mm]
All sequences of topologically ordered nodes are:
\begin{itemize}
\item A, B, C, D, E
\item A, B, C, E, D
\item B, A, C, D, E
\item B, A, C, E, D 
\end{itemize}
\item Maximal number of topologically ordered sequences is in a DAG with
no edges, in our case it consists of 5 isolated nodes. The number of
all topological orders is 120 in this case.
\item $n!$
\end{itemize}

\subsection*{Exercise 3}
$g.make$\\
$a:=g.new\_vertex(A)$\\
$b:=g.new\_vertex(B)$\\
$c:=g.new\_vertex(C)$\\
$d:=g.new\_vertex(D)$\\
$g.insert\_vertex(a)$\\
$g.insert\_vertex(b)$\\
$g.insert\_vertex(c)$\\
$g.insert\_vertex(d)$\\
$g.insert\_edge(g.new\_edge(1),a,b)$\\
$g.insert\_edge(g.new\_edge(3),b,b)$\\
$g.insert\_edge(g.new\_edge(7),c,d)$\\


\end{document}
