Here is another algorithm (due to J.\  B.\  Kruskal) to compute a minimum
spanning tree of a given graph~$G$.  The algorithm constructs a minimum
spanning tree $T$ by adding one edge at a time.  Initially, $T$ contains
all the vertices of the graph, but no edges.

\begin{itemize}

\item[(1)]  Let $T$ be the forest consisting of the vertices of $G$.

\item[(2)]  Find an edge $e$ of minimum weight in $G$.  Remove $e$ from
$G$.

\item[(3)]  If adding $e$ to $T$ does not create a cycle in $T$, then
add $e$ to $T$.

\item[(4)]Repeat  the previous step until there are no more edges in $G$.

\end{itemize}
%
Note that, during the execution of the algorithm, $T$ is a forest.
%
\begin{enumerate}

\item  Show the minimum spanning tree constructed by Kruskal's algorithm
for the graph in the second slide of page 123 of the lecture
notes. Give the sequence of edges added.

\item  Give an efficient way to perform Step (3).

\end{enumerate}
