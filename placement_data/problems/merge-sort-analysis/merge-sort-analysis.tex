The underlying idea of the mergesort algorithm for sorting a list L is as
follows:
\  \\
\  \\
\indent  (1) Sort the first half of the list.
\  \\
\indent  (2) Sort the second half of the list.
\  \\
\indent  (3) Merge the two sorted halves into one sorted list.
\  \\
\  \\
Let N be the number of elements in the list L and suppose T(N) is the time
required to sort L.  Also, assume N is a power of 2. Since the time necessary 
to sort N elements is T(N), the time necessary to sort N/2 elements must be 
T(N/2).  Thus, sorting half of the list L requires T(N/2) time.  From this it
follows that steps (1) and (2) from above require a total of 2T(N/2) time
for completion.  Step (3), merging two lists of size N/2 into one list of size
N, takes essentially N time.  So, steps (1), (2), (3) together can be
completed in 2T(N/2)+N time.  Therefore, mergesort must require 2T(N/2)+N
time for completion.  Since we previously let T(N) denote sorting time, we may
conclude that T(N)=2T(N/2)+N.  Given that this is the case and also that 
$T(1)=0$, solve for T(N)
explicitly (unwind the recurrence equation).  By doing so, you will also
be determining the time complexity of mergesort.






