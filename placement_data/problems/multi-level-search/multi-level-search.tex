Suppose we have some data stored in disk blocks. There are two kinds of
blocks in terms of their contents. One is data block, where the real
records are stored, another is index block, where we store some control
information. All the records within a data block are sorted. A record 
in index block consists of three fields: a key, an address of another
block and a boolean variable indicating whether the following block
contains data or indices. And the index records are also sorted on
the key within a index block. All the records whose key is between 
the key fields of two adjacent index records can be found at the block
pointed to by the address field of the latter one, regardless of the
type of that block. Design an efficient
algorithm to perform searching on this data structure and determine 
its time complexity. Suppose a block can contain $D$ data records 
or $I$ indice, and the total number of data {\em blocks} is $N$ which is
a power of $I$. Also, all the data blocks have the same length from
the root index block.
