A binary tree is said to be complete provided all non-leaf nodes in the tree
have two children and the paths to all leaves in the tree are of the same
length.
Let T be a complete binary tree.  Let X denote the sum of the path lengths to
all non-leaf nodes in the tree T.  Similarly, let Y denote the sum of the
path lengths to all leaves in T.  Supposing the number of non-leaf nodes is N,
determine constants a and b such that $Y=aX+bN$. (Hint: Draw a couple of
different complete binary trees to determine the relationship between
Y, X, and N in each.)
