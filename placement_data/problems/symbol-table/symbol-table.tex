Given your passion for computer science (and for finding a well-paying
job) it is expected that you will continue taking more CS courses.
Now, when you take a course on {\em compilers} next year, you will be
asked to implement such a creature.  As you may expect, this is not
trivial, so start thinking about it.  Now.  Good.  (See, we care about
you!)

Well, your compiler will read a file containing the source program in
some language, and will encounter variable names, function names, and
procedure names.  Let us call all of these {\em names}.  For each name,
you will need to keep a record with some information associated with it
in some {\em neat} data structure.  Now, every time a name is
encountered, you may need to:

\begin{enumerate}

	\item  Check to see if you did encounter it before. If you did
	not, you have to {\bf insert} it into your {\em neat} data
	structure.

	\item  If it is a name that you have seen before, {\bf retrieve}
	some of the values associated with it.

	\item  When you are changing scope by entering or exiting a
	function declaration, you will need to {\bf delete} some
	names.

\end{enumerate}

You have the following options for implementing your {\em neat} data
structure:  array, linked list, balanced binary tree and hash table.
Investigate the options by creating a table that gives for each
implementation, the expected time of the {\bf insert}, {\bf retrieve},
and {\bf delete} operations.  Explain your answers and discuss the
advantages and disadvantages of each implementation for the above name
handling application.
