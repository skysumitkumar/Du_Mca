Consider the following algorithm to simulate quicksort. 

1. If the sequence is empty or has only one element, then return the
   sequence. 

2. Otherwise, partition the array into three parts, \\
      pivot = first element. (say f)
   first part  : all elements < f
   second part : the element f (the pivot)
   third part  : all elements >= f

3. Now, recursively sort the first and the third parts (say we get
   back firstS, thirdS as the return values after
   sorting). Concatenate firstS, f, thirdS and return the sequence
   thus formed.

One can construct a {\it sort tree} for an execution of this
algorithm on an input string. Construct a {\it sort tree} for the
sequence {\tt (5 24 7 50 33 89 2 92 77)}

A {\it sort tree} for the input {\tt (5 4 6 7)} is shown below.

\begin{figure}[h]
\centerline{\psfig{file=new-simulate-qsort.eps}}
\caption{Sort tree for (5 4 6 7)}
\end{figure}






