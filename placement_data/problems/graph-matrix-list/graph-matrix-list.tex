We have introduced two data structures for representing a graph: {\em

adjacency matrix} and {\em adjacency lists} and we have said that
in some cases an adjacency matrix is preferable to adjacency lists
while in others it is not. This exercise asks you to choose
between them.

  Which data structure would you use
in each of the following cases?  Justify your choice.

\begin{enumerate}

\item  The graph has 10,000 vertices and 20,000 edges and it is important
to use as little space as possible.

\item  The graph has 10,000 vertices and 20,000,000 edges.

\item  You need to answer in constant time a query of the type:  ``Is
edge $(i,j)$ in the graph?''

\item  You need to answer in constant time a query of the type:  ``Is
there any vertex  connected to vertex $i$ or is $i$ isolated?''

\end{enumerate}

You can assume that the $N$ vertices are identified by integers in the
range $1 \cdots  N$.
