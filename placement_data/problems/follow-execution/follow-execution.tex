{\large\bfseries\itshape Follow the execution.} 
A {\it Sequence} is a container that stores its elements in a
linear order, which is imposed by the operations performed. Consider a
sequence $S$ of integers, which provides the following method (among
others).
\begin{itemize}
\item[]  \texttt{insertAtRank}($r$,$e$): inserts integer $e$ in the
  $r$-th position of $S$ and returns a locator to the newly inserted
  object.
\end{itemize}
Consider the following Java code :
\begin{verbatim}
for (int i=0; i<10 ; i++) {
    rank = (i/2)+1 ;

    S.insertAtRank(rank,i) ;

}
\end{verbatim}

Note here that ``/'' represents integer division.  I.e., only the
integral part of the quotient is considered.  Also, assume that \texttt{rank}
has earlier been defined to be an \texttt{int}.


\begin{enumerate}
\item  Show the sequence (just list its elements in order) after
each iteration of the {\tt for} loop. Assume that $S$ was initialized
to an empty sequence before the {\tt for} loop.

\item  Show the sequence after $n$ iterations of the {\tt

for} loop. (Hint : You might want to consider two cases, namely, $n$
being even and $n$ being odd)
\end{enumerate}

