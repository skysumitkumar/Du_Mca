When an adjacency matrix representation is used, most graph algorithms
require time $O(n^2)$ (where $n$ is the number of vertices), but there
are some exceptions. Here's one.

The race for the election of Brown's chairman is heating up and
$everyone$ at Brown is competing for it. Each candidate has a few
preferences (people who the person would be willing to accept as a
chairman). Of course, the set of preferences for a person includes
him/her self all the time.

What we are looking for is a ``perfect'' chairman who is in the set of
preferences of every person and who does not prefer anyone but him/her
self (wouldn't that make a good chairman?). In fact, all we want to
know whether such a person exists or not. Otherwise, we're willing to
live in anarchy.

Define a directed graph with the set of candidates as the vertices and
a directed edge from vertex $a$ to vertex $b$ if and only if $b$ is in
the set of preferences of $a$. 

Suppose that the number of people (also the number of candidates) is
$n$ Give an algorithm which executes in $O(n)$ time and determines if
such a perfect chairman exists or not. Assume that you are given the
graph described above in the form of an $n \times  n$ adjacency matrix.

% Show that determining whether a directed graph contains a {\it sink}-
% a vertex with indegree $|V|-1$ and out-degree 0- can be determined in
% time $O(V)$, even if an adjacency matrix representation is used.
