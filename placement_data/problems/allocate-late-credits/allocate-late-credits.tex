{\large\bfseries\itshape Allocation of late credits to programs.} 
We assume all of you have read and understood the course grading
policy This question concerns the grading policy and might help you in
improving your grade for the course! Read on \dots

Each of the 5 programs given out carries the same weight. Roberto, who
has taken the course, messes up his assignments so badly that he ends
up being a total of 170 hours late in submitting his programs. 
Even the 120 late credits that we have given him are not enough! His
only option is to distribute the late credits so that his penalty is
minimized. The following table is a description of the status of his
programs and the grades he receives on them.  \\

% The same grading policy was used last year and we gave out 5 programs
% for the semester, each program carrying the same weight for the
% overall grade. Roberto, who took the course, messed up his assignments so
% badly that he ended up being more than 120 hours late (on the whole)
% in submitting his programs. Hence, even the 120 late credits
% that we gave him were not enough! His only option was to distribute
% the late credits so that his penalty was minimized. 

% insert table here.
\begin{center}
\renewcommand{\arraystretch}{1.2}
\begin{tabular}{|c|c|c|c|}
\hline
Program & Grade               & Late by   & Late credits    \\
\#            & (on a scale of 10)  & (hours)   & (one possible allocation) \\    \hline
\hline
1  &  5 & 40 & 40 \\  \hline
2  & 10 & 40 & 40 \\  \hline
3  & 10 & 50 &  0 \\  \hline
4  &  5 & 40 & 40 \\  \hline
5  & 10 &  0 &  0 \\  \hline
Total & 40 & 170 & 120 \\
\hline

\end{tabular}
\end{center}
\vspace{2mm}

Let us call the total penalty that he incurs because of the programs
being submitted late as the \emph{late penalty}.

\begin{enumerate}
\item  One way of allocating the credits would be to see that he gets
  zero penalty on the maximum number of programs.  Following this
  principle led him to the allocation shown in the last column of the
  table. Compute his late penalty if this allocation were
  followed.  Also, compute his total score on the 5 programs (equal to
  $40$ minus the late penalty).
  
\item  What would his late penalty and total score be if he were
  to take 10 late credits off program 4 and give them to program 3?
  
\item  What do you think is the best way for him to allocate the late
  credits?  Outline a scheme of allocation of late credits so that for
  any general distribution of late hours for the above 5 programs,
  your scheme will minimize the late penalty that Roberto
  incurs.
\end{enumerate}


