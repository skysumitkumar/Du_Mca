Assume $A$ is an array of $N$ distinct integers indexed from $1$ to
$N$.  Also, let $B$ be an auxiliary array (work space) of the same type
as $A$.  Consider the following sorting algorithm:

\begin{tt}
\begin{tabbing}
for $i:=1$ to $N$ do \\
beg\=in  \\
      \>  $x:=1$ \\
      \>  for \=  $j:=1$ to $N$ do \\
      \>        \>  if \=  $A[j]>A[i]$ then \\
      \>        \>        \>  $x:=x+1$ \\
      \>  $B[x]:=A[i]$ \\
end \\
for $i:=N$ downto $1$ do \\
      \>  $A[N-i+1]:=B[i]$
\end{tabbing}
\end{tt}

Explain how the algorithm sorts array $A$.  Also, give the time
complexity of the algorithm.
