In a color digitized image, a pixel is represented by a string of bits
that indicates its color.  For example, assume that 16 colors are
available, so that each color is represented by a string of 4 bits.
Let {\tt 1010} stand for blue, and {\tt 1111} stand for red.  Four
consecutive pixels colored blue, blue, blue and red, respectively, will
be represented by the bitmap {\tt 1010 1010 1010 1111}. Suppose now
that we have an image with more than half of its pixels in a certain
background color.  Because the value of that background color can be
arbitrary, using run-length encoding to compress the bitmap will not be
efficient.  Find a simple ``trick'' to transform the original bitmap
into another one on which run-length encoding will be much more
efficient.
