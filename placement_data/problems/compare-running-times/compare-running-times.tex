{\large\bfseries\itshape Comparison of running times.} 
This question is to give one an idea of the rate at which functions
grow. For each function $f(n)$ and time $t$ in the following table,
determine the largest  size $n$ of a problem that can be solved in time
$t$, assuming that the algorithm to solve the problem takes $f(n)$
microseconds. Note that $\log  n$ means the logarithm in base~2 of $n$.
Some entries have already been filled to get you started.

% insert the table here ...
\begin{center}
\renewcommand{\arraystretch}{1.4}
\begin{tabular}{|c|c|c|c|c|}
\hline
                  & 1 Second & 1 Hour & 1 Month & 1 Century \\  \hline  \hline
$\log  n$ & $\approx  10^{300000}$  &      &       &         \\  \hline
$\sqrt  n$ &       &      &       &         \\  \hline
$n$      &        &      &       &         \\  \hline
$n\log  n$ &       &      &       &         \\  \hline
$n^2$    &        &      &       &         \\  \hline
$n^3$    &        &      &       &         \\  \hline
$2^n$    &        &      &       &         \\  \hline
$n!$     &        &  12  &       &         \\  

\hline
\end{tabular}
\end{center}


