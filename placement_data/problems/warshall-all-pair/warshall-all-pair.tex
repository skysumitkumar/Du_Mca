{\em Warshall's Algorithm}, introduced in class, computes the transitive
closure of a digraph. If the digraph is represented by an $N \times  N$
adjacency matrix $A$, the algorithm can be compactly described as
follows:

\begin{verbatim}
    for k := 1 to N
        for i := 1 to N
            for j := 1 to N
                A[i,j] := A[i,j] OR (A[i,k] AND A[k,j])
\end{verbatim}

The values stored in $A$ are boolean values ({\tt TRUE} or {\tt

FALSE}).  If there is an edge from vertex $i$ to vertex $j$, $A[i,j]$ =
{\tt TRUE}; otherwise, $A[i,j]$ = {\tt FALSE}.  The time complexity is
$O(N^3)$.

\indent
Suppose we have a weighted digraph that is represented by its
adjacency matrix and by a weight matrix $W$ such that

\begin{itemize}

\item  $W[i,j] = 0$ if $i=j$;

\item  $W[i,j] = \ell$  if $i\neq  j$ and there is an edge with weight
$\ell$  from $i$ to $j$; and

\item  $W[i,j] = \infty$  if $i\neq  j$ and there is no edge from  $i$ to
$j$.

\end{itemize}

Modify Warshall's algorithm by adding one or more instructions, such
that when the computation terminates, $W[i,j]$ stores the length of a
shortest path from vertex $i$ to vertex $j$.
