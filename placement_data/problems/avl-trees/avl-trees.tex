In 1962, G.M. Adelson-Velskii and E.M. Landis proposed a scheme for 
``balancing''
binary search trees.  Named in their honor, ``an {\em AVL Tree} is a binary
search tree in which the heights of the left and right subtrees of the root
differ by at most 1 and in which the left and right subtrees are again
AVL trees''. (Robert L. Kruse, {\em Data Structures and Program Design},
Prentice Hall, 1987, 344-345)  Assuming that $T_{1}$, $T_{2}$, $T_{3}$, and
$T_{4}$ are AVL trees of height $h$, alter the following binary search tree
to yield an AVL tree (Kruse, 349).

\begin{figure}[h]
\begin{picture}(300,200)

\put(170,150){\circle{20}}
\put(164,147){$K_{1}$}
\put(163,143){\line(-3,-1){50}}
\put(177,143){\line(3,-1){50}}
\put(100,87){\framebox(15,40)}
\put(103,103){$T_{1}$}
\put(237,120){\circle{20}}
\put(231,117){$K_{2}$}
\put(230,113){\line(-2,-1){25}}
\put(244,113){\line(2,-1){30}}
\put(204,90){\circle{20}}
\put(198,87){$K_{3}$}
\put(197,83){\line(-1,-1){15}}
\put(211,83){\line(1,-1){15}}
\put(180,27){\framebox(15,40)}
\put(183,43){$T_{2}$}
\put(220,27){\framebox(15,40)}
\put(223,43){$T_{3}$}
\put(267,57){\framebox(15,40)}
\put(270,73){$T_{4}$}

\end{picture}\end{figure}



