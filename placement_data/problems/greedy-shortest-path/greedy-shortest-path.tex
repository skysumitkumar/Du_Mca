Consider the following greedy strategy for finding a shortest path
from vertex {\em start} to vertex {\em goal} in a given connected graph.

\begin{enumerate}
  
\item  [1:] Initialize {\em path} to {\em start}.
  
\item  [2:] Initialize {\em VisitedVertices} to \{{\em start}\}.
  
\item  [3:] If {\em start=goal}, return {\em path} and exit.
  Otherwise, continue.
  
\item  [4:] Find the edge ({\em start,v}) of minimum weight such that
  {\em v} is adjacent to {\em start} and {\em v} is not in {\em

    VisitedVertices}.
  
\item  [5:] Add {\em v} to {\em path}.
  
\item  [6:] Add {\em v} to {\em VisitedVertices}.

\item  [7:] Set {\em start} equal to {\em v} and go to step 3.

\end{enumerate}

Does this greedy strategy always find a shortest path from {\em start}
to {\em goal}?  Either explain intuitively why it works, or give a
counter-example.

