The application process at Tamarindo University is as follows:

\begin{itemize}

	\item  Each applicant takes a standard test consisting of eight
	subjects: {\em English} (E), {\em Math}
	(M), {\em Physics} (P), {\em Chemistry} (C), {\em Biology} (B),
	{\em History} (H), {\em Geography} (G), and {\em Sociology}
	(S).

	\item  Each applicant also submits a list of departments where
	he/she wants to be admitted, ranked by preference.

	\item  Each department $D$ can admit at most $A_D$ students, and
	ranks each applicant $a$ using a function $\sigma_D(a)$  that
	computes a weighted average of the test scores obtained
	by~$a$.  The function $\sigma_D$  is different for each
	department.  For example, the Computer Science Department uses
	the function

	\[
	\sigma_{\rm CS}(a) = 0.1 E(a) + 0.3 M(a) + 0.2P(a) + 0.1 C(a) +
	0.1 B(a) + 0.05 H(a) + 0.1 G(a) + 0.05 S(a) .
	\]

\end{itemize}

An admission blunder occurs when an applicant $a$ is admitted into
department $D$ but
%
\begin{itemize}

\item  $a$ prefers department $E$ over $D$; and

\item  either fewer than $A_{E}$ students have been admitted into $E$,
or there is another applicant $b$ admitted into $E$ such that
 $\sigma_{E}(b) < \sigma_{E}(a)$.

\end{itemize}
%
Clearly, $a$ should have been admitted into $E$.

\medskip

Describe an efficient admission algorithm that prevents blunders.  What
is the time complexity of your algorithm?  Assume there are $N$
applicants and $M$ departments.
