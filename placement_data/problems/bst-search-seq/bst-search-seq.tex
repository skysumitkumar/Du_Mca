In the search for an element $x$ in a binary tree, the list of nodes
examined (i.e that $x$ is compared to) is called the {\em Node

Sequence} for the search. For example, the {\em Node Sequence} for the
search shown in page 160 of the lecture slides (the slide with the
heading {\em Binary Tree : How to Search}) is M,H,J,I.

Suppose that we have the numbers 1 to 1000 in a binary search tree and
we want to search for the number 363. Which of the following sequences
{\it could not} be the {\em Node Sequence} for the search? Explain why.

\begin{enumerate}
\item  2, 252, 401, 398, 330, 344, 397, 363.
\item  924, 220, 911, 244, 898, 258, 362, 363.
\item  925, 202, 911, 240, 912, 245, 363.
\end{enumerate}
