The algorithm explained in class for computing a closest pair among $N$
points in the plane has time complexity $O(N \log  N)$, but is somewhat
complicated.  Consider now a set of $N$ points in one dimension (i.e.,
$N$ points on a line).  Clearly, the general algorithm can also be
applied to this problem.  However, we should be able to take advantage
of the fact that all the points are on the same line.  Can you find an
algorithm for computing a closest pair of points in one dimension that
is {\em simple} and runs in $O(N \log  N)$ time?
