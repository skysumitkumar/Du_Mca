Given a bipartite graph $G=(V,E)$ and a matching $M$ of $G$, an edge $e\in  E$
is said to be {\em matched} if it is contained in $M$.  Otherwise, the edge is
{\em free}.  We define a simple path in $G$ as {\em alternating} if
its edges strictly alternate between free and matched.  Suppose $G$ 
contains an alternating path where both the first and last edges on the path 
are
free.  Also, assume that no matched edges are incident on the ``endpoint'' 
vertices of the path.  With this given, prove that
$M$ is not a maximum matching.  Hint: consider the free edges in the
path.
