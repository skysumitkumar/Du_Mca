You are a member of a team that designs the next generation
mars-mobile, to be used in the exploration of the famous martian region
``Hellas'' (named after ancient Greece).  The characteristic of this
region is that it is relatively smooth, resembling some lunar regions
(like ``mare serenitionalis,'' where Apollo~11 landed), but with many
little unstable craters --- which must be avoided.  Your mars-mobile
will use a map, generated by the programs of the vision group. In this
map the craters are represented as large black dots like this one:
(This is the only area the mars-mobile can see.  It is unsafe to try to
move outside of the region enclosed by these points.)

\input{psfig} \vspace{4mm} \par
\begin{figure}[h]
\centerline{\psfig{file=fi-voronoi-cellular-phone.ps}}\end{figure}
 \par
\vspace{4mm}


Your group's task is to  plan out all the possible routes that the mars
mobile can take safely while crossing this area.  The safety
restrictions require you to avoid coming close to any crater as much as
possible. In other words, if you want to cross the area between two
craters $c_1$ and $c_2$, you should move on the perpendicular bisector
of $(c_1, c_2)$.

Draw the possible routes in the above map.  What is the geometric
object that represents these routes?  What are some of the suggested
routes if your mars-mobile wants to move from north to south?
