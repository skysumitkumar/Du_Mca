Providence Rent-a-Wreck wants to provide computerized driving
directions to its customers.  The driving directions are a
sequence of statements of the type:

\begin{quote}
At {\em location-name}, go {\em direction} on {\em street-name}.
\end{quote}

Their computer stores the map of the city as follows:

\begin{itemize}

\item  There are $N$ locations, each with a unique integer ID in the
range $1 \cdots  N$.  The locations include all the street
intersections, plus various important buildings, such as the CIT.  An
array {\sf Location} of size $N$ stores in element {\sf Location}$[i]$
the name of the location with ID~$i$.  Also, a red-black tree stores
the locations alphabetically sorted by name, where each internal node
stores a location and its~ID.

\item  An $N \times  N$ 2-dimensional array {\sf Connect} stores info on
streets, as follows.  If there is a street $s$ that allows one to drive
from location $i$ to location $j$ along $s$, then {\sf Connect}$[i,j]$
stores the name of $s$ and the direction of travel from $i$ to $j$
along $s$ (N, S, E, or W).  If two or more such streets exist, an
arbitrary one of them is chosen. Otherwise (there is no street that
allows one to drive from $i$ to $j$), {\sf Connect}$[i,j]$ stores the
conventional symbol~\#.

\end{itemize}

Give an efficient algorithm that given the above data structure and the
names of two locations, $A$ and $B$, prints directions to drive from
$A$ to $B$ going through the minimum number of streets.

