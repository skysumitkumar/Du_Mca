Consider the following pseudocode for Quicksort (taken from the Class Notes):
\begin{verbatim}
procedure Quicksort(left, right:integer);
   index: integer;

   if (right - left <= 1) then return;
   if (right > left) then
     index := Partition(left, right);
     Quicksort(left, index-1);
     Quicksort(index+1, right);
end Quicksort;


function Partition(left, right);
   p : integer;

   p := A[right];
   scan from left until A[i] > p is found;
   scan from right until A[j] < p is found;
   exchange A[i] and A[j];
   continue scanning and exchanging until pointers i, j, cross;
   exchange p with A[i];
   return (p);
end Partition;
\end{verbatim}

Show how the following array elements would be processed when we apply the
given psudocode:

$$6,\hbox{      } 2,\hbox{      } 12,\hbox{      } 5,\hbox{      } 15,\hbox{      } 9,\hbox{      } 7$$

Show your intermediate steps (i.e. draw the array immediately after any elements
are exchanged, and indicate where the $i$ and $j$ pointers are pointing at that
time).
