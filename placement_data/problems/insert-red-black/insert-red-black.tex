Convert the red-black tree under title {\em A Super Example} on 
page 52 of lecture notes into its equivalent 2-3-4 tree. Then
insert key {\em S} into this 2-3-4 tree, and adjust it such that
all the paths from root to leaf have the same length. Finally convert
this 2-3-4 tree after insertion back into its equivalent red-black
tree. Check to see that if it's the same as the one appeared on
lecture notes. Do you think this is an easier way to perform
insertion on red-black trees? (Be sure that you have drawn out
all the stages during these conversions.)
