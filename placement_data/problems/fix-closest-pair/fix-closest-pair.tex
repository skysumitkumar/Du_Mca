In the divide and conquer algorithm presented in class for finding the 
closest pair of points in a given set, the
{\em Euclidean distance} is used to
compute the distance between two points $p_{1}$ and $p_{2}$.  That is,
$d(p_{1},p_{2})=\sqrt{(x_{2}-x_{1})^2+(y_{2}-y{_1})^2}$ where
$p_{1}=(x_{1},y_{1})$ and $p_{2}=(x_{2},y_{2})$.  This is the standard notion
of distance between points.  However, we may also define distance in other
ways.

\begin{enumerate}

\item  Suppose we define $d(p_{1},p_{2}) = |x_{2}-x_{1}| +
|y_{2}-y_{1}|$ (this is called {\em Manhattan distance}).  Does the
divide and conquer algorithm still work?

\item  Suppose we define $d$ such that $d(p_{1},p_{2})=|y_{2}-y_{1}|$.
(In practice we may define distance as such, for instance, to find the
smallest difference between stock market closings over a given number
of days, months, or years).  Does the divide and conquer algorithm
still work?

\end{enumerate}
