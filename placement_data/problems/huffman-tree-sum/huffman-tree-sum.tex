In a Huffman encoding tree, all the leaf nodes are called external nodes
and the others are called internal nodes. The tree has the property that 
the value of an internal node is the sum of the values of it children,
and the value of an external node is the frequency of the character
associated with this node. We can expect the length of the compressed
text to be the sum of the products of the value and the depth (the depth
of the root is 0) of each external node, this is called the {\em weighted

external path length}. There's an interesting property that this length
is exactly equal to the sum of the values of all the internal nodes.
For instance, in the following tree, the weighted external path length
is $5 \times  1 + 2 \times  3 + 2 \times  3 + 1 \times  3 + 1 \times  3 = 23$,
and the sum of the values of the internal nodes is $11 + 6 + 4 + 2 = 23$.
Give a mathematical proof to that. ({\em Hint:} Prove that it holds true
for the tree with only one internal node at first. Then assume that
it holds true for a tree containing $n$ internal nodes, try to prove it
for a tree with $n+1$ internal nodes. Choose a node with two external-node
children and consider it as an external node. Now the problem goes back
to the case of a tree with $n$ internal nodes.)

\setlength{\unitlength} {0.15in}

\begin{figure}[h]
\begin{picture}(17,9)(0,0)

\put(2,8) {\circle{1}}
\put(2,8) {\makebox(0,0) {$11$}}
\put(1.50,7.50) {\line (-3,-2){1.80}}
\put(2.50,7.50) {\line (6,-1){7.57}}

\put(0,6) {\framebox(0,0) {$5$}}

\put(10,6) {\circle{1}}
\put(10,6) {\makebox(0,0) {$6$}}
\put(9.50,5.50) {\line (-4,-1){3.64}}
\put(10.50,5.50) {\line (4,-1){3.64}}

\put(6,4) {\circle{1}}
\put(6,4) {\makebox(0,0) {$4$}}
\put(5.50,3.50) {\line (-3,-2){1.80}}
\put(6.50,3.50) {\line (3,-2){1.80}}

\put(4,2) {\framebox(0,0) {$2$}}

\put(8,2) {\framebox(0,0) {$2$}}

\put(14,4) {\circle{1}}
\put(14,4) {\makebox(0,0) {$2$}}
\put(13.50,3.50) {\line (-3,-2){1.80}}
\put(14.50,3.50) {\line (3,-2){1.80}}

\put(12,2) {\framebox(0,0) {$1$}}

\put(16,2) {\framebox(0,0) {$1$}}

\end{picture}\end{figure}

