Consider the following red-black tree.

\begin{figure}[h]
\begin{picture}(200,250)

\put(200,200){\circle{30}}
\put(198,198){O}
\thicklines
\put(188,188){\line(-3,-2){32}}
\put(187,189){\line(-3,-2){32}}
\put(186,190){\line(-3,-2){32}}
\put(212,188){\line(3,-2){32}}
\put(213,189){\line(3,-2){32}}
\put(214,190){\line(3,-2){32}}
\thinlines
\put(146,153){\circle{30}}
\put(144,151){B}
\thicklines
\put(134,141){\line(-3,-2){43}}
\put(133,142){\line(-3,-2){44}}
\put(132,143){\line(-3,-2){45}}
\thinlines
\put(158,141){\line(1,-3){6}}
\put(76,106){\framebox(15,7)}
\put(164,106){\circle{30}}
\put(162,104){N}
\thicklines
\put(152,94){\line(-3,-2){43}}
\put(151,95){\line(-3,-2){44}}
\put(150,96){\line(-3,-2){45}}
\put(176,94){\line(1,-3){9}}
\put(177,95){\line(1,-3){9}}
\put(178,96){\line(1,-3){10}}
\thinlines
\put(94,59){\framebox(15,7)}
\put(182,59){\framebox(15,7)}
\put(254,153){\circle{30}}
\put(252,151){U}
\put(242,141){\line(-1,-3){6}}
\put(266,141){\line(3,-2){32}}
\put(230,106){\circle{30}}
\put(228,104){R}
\thicklines
\put(218,96){\line(-1,-3){10}}
\put(217,97){\line(-1,-3){10}}
\put(216,97){\line(-1,-3){10}}
\put(215,98){\line(-1,-3){11}}
\put(242,94){\line(1,-3){9}}
\put(243,95){\line(1,-3){9}}
\put(244,95){\line(1,-3){9}}
\put(245,97){\line(1,-3){10}}
\thinlines
\put(204,59){\framebox(15,7)}
\put(250,59){\framebox(15,7)}
\put(310,106){\circle{30}}
\put(308,104){W}
\thicklines
\put(298,96){\line(-1,-3){10}}
\put(297,97){\line(-1,-3){10}}
\put(296,97){\line(-1,-3){10}}
\put(295,98){\line(-1,-3){11}}
\put(322,94){\line(3,-2){42}}
\put(323,95){\line(3,-2){42}}
\put(324,95){\line(3,-2){42}}
\put(325,96){\line(3,-2){45}}
\thinlines
\put(284,59){\framebox(15,7)}
\put(364,59){\framebox(15,7)}

\end{picture}\end{figure}


Insert key {\tt F}.  Draw all intermediate trees in the rebalancing.
