A very natural way of representing a relation $R$ on a set $S$ is with a
directed graph $G$, where each element of $S$ has a corresponding node in $G$
and each element of $R$ corresponds to a directed edge in $G$.  For instance,
if $(a,b)\in  R$ for $a,b \in  S$, then $a$ and $b$ are nodes in $G$, and there
is a directed edge from $a$ to $b$.  Assume $R$ is transitive and also that
for each $x\in  S$, there is an element $y\in  S$ such that $(x,y)\in  R$.
What condition then holds in the digraph corresponding to $R$?  Justify your
answer.
